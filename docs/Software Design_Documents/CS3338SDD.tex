
\documentclass[11pt]{article}
\usepackage[margin=1in]{geometry}
\usepackage{longtable}
\usepackage{hyperref}
\usepackage{titlesec}
\usepackage{fancyhdr}
\usepackage{graphicx}

\pagestyle{fancy}
\fancyhf{}
\rhead{CS3338 Final - LACPD1}
\lhead{Software Design Document}
\rfoot{Page \thepage}

\titleformat{\section}{\large\bfseries}{\thesection}{1em}{}
\titleformat{\subsection}{\normalsize\bfseries}{\thesubsection}{1em}{}

\title{Software Design Document (SDD) \\
CS3338 Final Based On LACPD1: AI-Powered Legal Analysis of Police Misconduct}
\author{Group 1: Brian Andrade, Christian Garcia, Haonan Ma, Jared Martinez, Alejandro Vargas}
\date{Version 0.6 \\ 5/09/2025}

\begin{document}
\maketitle
\thispagestyle{empty}
\newpage

\tableofcontents
\thispagestyle{empty}
\newpage

\section*{Version History Table Example}
\begin{longtable}{|l|l|p{8cm}|l|}
\hline
\textbf{Version} & \textbf{Date} & \textbf{Comment} & \textbf{Author(s)} \\
\hline
0.1 & 04/-/2025 & Initial project planning, selected LACPD1 baseline & Name \\
0.2 & 04/-/2025 & Drafted base architecture and workflow adaptation & Name \\
0.3 & 05/-/2025 & Snapshot 1: added ...  & Name \\
0.4 & 05/-/2025 & Snapshot 2: added ...  & Name \\
0.5 & 05/-/2025 & Snapshot 3: added ...  & Name \\
0.6 & 05/-/2025 & Snapshot 4: Finalized version + glossary & Name \\
\hline
\end{longtable}

\newpage
\section{Introduction}
\subsection{Purpose}
This Software Design Document (SDD) presents the technical blueprint for a web-based system inspired by LACPD1. It aims to streamline the legal review process of police misconduct cases by leveraging cloud services and artificial intelligence. Our system simulates the original LACPD1 solution, focusing on document ingestion, natural language analysis, and semantic search, all while maintaining scalability and data security. The SDD ensures all contributors are aligned on architectural vision, software behavior, and integration goals.

\subsection{Intended Audience}
\begin{itemize}
\item Developers – for implementation and debugging
\item Project Managers – to oversee milestones and scope alignment
\item Stakeholders (e.g., legal tech educators) – to understand solution impact
\item Testers – to design validation strategies and test cases
\item Documentation Writers – to write end-user and technical manuals
\end{itemize}

\subsection{System Overview}
The system is designed to allow users to upload legal transcripts, which may be scanned images or PDFs. These documents are processed via OCR (Optical Character Recognition) and stored securely. Using NLP (Natural Language Processing), the text is tokenized, chunked semantically, and enriched with metadata such as officer names, badge numbers, and case dates. All content is vectorized and indexed, allowing users to query documents via a semantic search interface or explore system-flagged inconsistencies. The goal is to support legal teams in reviewing, auditing, and comparing potential misconduct across transcripts.

\newpage
\section{System Architecture}
\subsection{Workflow and Site Breakdown}
\begin{itemize}
\item \textbf{Transcript Upload:} Users securely upload legal transcripts via the web interface.
\item \textbf{OCR Processing:} Scanned or image-based documents are converted into machine-readable text using OCR.
\item \textbf{Preprocessing:} The text is cleaned, tokenized, and split into meaningful segments.
\item \textbf{Entity Recognition:} NLP techniques identify officer names, badge numbers, locations, and relevant legal entities.
\item \textbf{Metadata Enrichment:} Extracted data is tagged and appended to transcripts.
\item \textbf{Semantic Embedding:} Documents are vectorized using sentence embeddings and stored in a FAISS index.
\item \textbf{RAG-Based Q\&A:} Users can submit legal queries and receive contextually relevant answers powered by retrieval-augmented generation (e.g., Amazon Titan, Claude).
\item \textbf{Visualization:} Summarized results are displayed in an interactive frontend with highlighting and export features.
\end{itemize}

\subsection{Component Diagram}
\textbf{Included Technologies:}  
FastAPI, Next.js, TailwindCSS, LangChain, Bedrock (Titan \& Claude), FAISS, AWS S3, Box.com, Salesforce API

\newpage
\section{User Interface}
\subsection{How to Use}
Upon logging in, users can:
\begin{itemize}
\item Upload transcripts by dragging and dropping files or using the file selector.
\item Review analysis results such as flagged inconsistencies and named entities.
\item Run questions through the Q\&A system to understand case relevance.
\item Download annotated versions of documents or export results to CSV.
\end{itemize}

\subsection{Database Explanation}
\begin{itemize}
\item \textbf{Box.com:} Stores the original and annotated files, linked with metadata fields such as date, agency, and officer name.
\item \textbf{AWS S3:} Backup storage for all processed documents and vectorized chunks.
\item \textbf{FAISS:} A high-performance similarity search index for enabling semantic retrieval.
\item \textbf{Salesforce:} Integrates case management data, ensuring alignment between uploaded evidence and client records.
\end{itemize}

\newpage
\section{Glossary}
\begin{longtable}{|l|p{10cm}|}
\hline
\textbf{Term} & \textbf{Definition} \\
\hline
OCR & Optical Character Recognition – converting images of text into machine-readable text. \\
RAG & Retrieval-Augmented Generation – combining external document retrieval with LLMs. \\
LLM & Large Language Model – a generative language model like Claude or Titan. \\
NER & Named Entity Recognition – an NLP process to detect people, places, dates, and organizations. \\
FAISS & Facebook AI Similarity Search – vector search engine for fast similarity-based retrieval. \\
S3 & AWS Simple Storage Service – cloud-based scalable file storage. \\
\hline
\end{longtable}

\newpage
\section{References}
\begin{itemize}
\item \url{https://ascent.cysun.org/project/project/view/233}
\item \url{https://ascent.cysun.org/project/project/search?searchText=LA%20County%20Public%20Defender%27s%20Office}
\item \url{https://docs.aws.amazon.com/}
\item \url{https://developer.box.com/}
\item \url{https://docs.langchain.com/docs/}
\item \url{https://docs.aws.amazon.com/bedrock/latest/userguide/what-is-bedrock.html}
\item SDD LACPD1
\item SRD LACPD1
\item CS4691 Fall 2024 AWS Team Presentation File
\item CS4692 Spring 2025 AWS Team Presentation File
\end{itemize}

\end{document}
