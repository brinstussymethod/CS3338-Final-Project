\documentclass[12pt]{article}
\usepackage[utf8]{inputenc}
\usepackage{geometry}
\geometry{margin=1in}
\usepackage{hyperref}
\usepackage{graphicx}
\usepackage{longtable}
\usepackage{fancyhdr}
\usepackage{titlesec}



\pagestyle{fancy}
\fancyhf{}
\rhead{LACPD1 AWS SRS}
\lhead{Version 2.5}
\cfoot{\thepage}

\begin{document}
\begin{titlepage}
    \centering
    \vspace*{2cm}

    {\Huge\bfseries Software Requirements Specification (SRS)\\
    for\\
    LACPD1 AWS\\}

    \textbf{Version 2.5}\\
    05/08/25\\

    \textbf{Prepared by:}\\
    Brian Andrade, Christian Garcia, Haonan Ma, Jared Martinez, Alejandro Vargas

    \textbf{Sponsored by:}\\
    Los Angeles County Public Defender's Office\\

    \vfill
    {\large \today}
\end{titlepage}

\tableofcontents
\setcounter{tocdepth}{3}  


\newpage

\section{Introduction}
\subsection{Purpose}
The Software Requirements Specification (SRS) document describes a transcript analysis
 and case management system developed for the Los Angeles County Public Defender’s Office
 (LACPD). This system is specifically designed to analyze court transcripts and flag police
 misconduct patterns. The system will integrate AWS cloud services and advanced data science
 techniques, particularly Natural Language Processing (NLP) and generative AI, to detect
 potential misconduct, identify key entities (such as officer names and badge numbers), and flag
 inconsistencies in testimonies for legal use.
 The SRS aims to ensure all stakeholders, including project sponsors, developers, testers,
 and legal personnel, have a shared understanding of the system’s capabilities and its role in
 supporting legal workflows.


\subsection{Scope}
 The Transcript Analysis and Case Management System is a software solution created for
 the Los Angeles County Public Defender’s Office (LACPD). Its main goal is to improve the
 efficiency and accuracy of legal workflows by enhancing the analysis of court transcripts. By
 utilizing AWS cloud services, Natural Language Processing (NLP), and generative AI models,
 the system provides real-time processing of transcripts, helping legal teams identify critical
 insights

\textbf{Key Benefits }
\begin{itemize}
  \item  Streamlined Workflows: Automates tasks like document review and analysis.
  \item Accurate Insights: Highlights critical information such as misconduct patterns and
 inconsistencies.
  \item  Scalable Design: Adapts to increasing case data without compromising performance.

\textbf{Objectives and Capabilities }
\begin{enumerate}

    \item Misconduct Detection: 
	\begin{itemize}
	 	\item Automatically identify instances of police misconduct based on flagged patterns or
			 keywords in transcripts and evidence documents.
	 	\item  Highlight repeated behaviors or systemic issues to strengthen arguments
	\end{itemize}
    \item Entity  Extraction:
	\begin{itemize}
		 \item Extract and categorize important entities such as officer names, badge numbers, dates,
	 		and locations.
	 	\item Present these entities in a structured, searchable format for quick reference
	\end{itemize}
    \item Testimony Analysis:
	\begin{itemize}
		\item Analyze witness statements and cross-reference them to identify contradictions or
 			inconsistencies.
		\item  Highlight tone shifts or emotional language that may impact case outcomes
	\end{itemize}
    \item Scalability and Performance:
	\begin{itemize}
		\item  Leverage cloud-based infrastructure to process and store large volumes of legal
 			documents.
		\item Maintain low latency for real-time document retrieval and analysis, even with complex
  			cases.
	\end{itemize}
\end{enumerate}


\textbf{Applications and Benefits }

 \textbf{Applications}
\begin{enumerate}
    \item Legal Case Preparation:
	\begin{itemize}
	 	\item  Streamlines the review of transcripts and documents, allowing attorneys to focus on
 			building arguments
	\end{itemize}
    \item  Legal Research:
	\begin{itemize}
		 \item Assists in identifying patterns or precedents relevant to ongoing cases
	\end{itemize}
\end{enumerate}

\textbf{Benefits }
\begin{itemize}
  \item   TimeEfficiency: Automates repetitive tasks, reducing manual effort
  \item  ImprovedAccuracy: Minimizes human error by flagging critical issues
  \item  EnhancedInsights: Provides actionable information that strengthens case strategies
 \item  Scalability: Accommodates growing data demands and user bases seamlessly
\end{itemize}


 \textbf{Exclusion:}
\begin{enumerate}
    \item Judgment Prediction:
	\begin{itemize}
	 	\item   The software does not predict case outcomes or court decisions
	\end{itemize}
    \item Complex Legal Interpretation:
	\begin{itemize}
		 \item  It does not replace attorney expertise in interpreting legal arguments or evidence
	\end{itemize}
    \item  Manual Data Input:
	\begin{itemize}
		 \item   Requires users to upload documents and provide metadata manually; it cannot fetch data
 from external sources
	\end{itemize}
    \item  Cross-Language Analysis:
	\begin{itemize}
		 \item   Currently limited to documents in English; non-English transcripts require
 pre-translation.
	\end{itemize}
\end{enumerate}
\end{itemize}




\subsection{Overview}
 This SRS document is organized as follows:
\begin{itemize}
  \item Section 1: Provides an introduction, including the purpose, scope, and structure of the
 document
  \item  Section 2: Describes the general factors affecting the software, including the product
 perspective, product functions, and user characteristics
  \item  Section 3: Details the specific software requirements, including functional requirements,
 external interfaces, and performance criteria
  \item Section 4: Lists any references to external documents or standards used in this specification
\end{itemize}

\subsection{Definitions, Acronyms, and Abbreviations}
 There will be more in Appendix A: Glossary

\subsection{References}
\begin{itemize}
  \item  Provide a complete list of all documents referenced elsewhere in the SRS;
  \item  Identify each document by title, report number (if applicable), date, and publishing
 organization
\end{itemize}

\section{General Description}
\subsection{Product Perspective}
 The LACPD Transcript Analysis System is a cloud-based software solution developed as a
 standalone application that integrates into LACPD’s existing case management workflow. The
 system will utilize Box.com for data storage and AWS cloud services for computing power and
 machine learning model hosting. It will use Natural Language Processing (NLP) models to
 extract key information and classify different segments of the court transcripts. The product will
 also provide a web-based interface for querying and visualizing the relationships between
 various cases

 
\begin{itemize}
    \item \textbf{System Interfaces:}
	\begin{itemize}
	 	\item   Thesystem interacts with LACPD's case management processes by enabling the manual
 upload and export of relevant case data, such as court documents and transcripts. While
 direct integration through APIs with LACPD’s existing systems is currently dependent on
 their approval and infrastructure capabilities, the system is designed to be
 integration-ready. This allows for future API-based synchronization, should LACPD
 permit direct data exchange. Exported insights and flagged data can be provided in
 widely compatible formats (e.g., JSON, XML, CSV) for manual or automated
 re-integration into external tools. Key data such as case details, court record hearings,
 transcripts, and other relevant documents will be exchanged, reducing the need for
 manual input.
	\end{itemize}
    \item \textbf{User Interfaces}
	\begin{itemize}
		 \item  The user interface is designed to be minimalistic, clean, and easy to use
		 \item  Being intuitive, it allows LACPD personnel to easily utilize and navigate data
 		\item TheUIwill allow users to query court transcripts, view analysis results, and explore
			 visualized data relationships through web-based dashboards
	\end{itemize}
    \item  \textbf{Hardware Interfaces:}
	\begin{itemize}

		\item Requires minimal hardware. Users will access the system through standard
			 internet-connected devices
		\item  Functionality will occur on all screen sizes and device types while being optimized for low resources
	\end{itemize}
    \item  \textbf{Software Interfaces}
	\begin{itemize}
		 \item    Utilizes standard libraries for NLP tasks and may integrate with other LACPD software tools.
		\item Wemayinclude libraries such as spaCy or NLTK for text processing and TensorFlow or PyTorch for machine learning tasks.
		\item  Thesoftware will also handle basic data formats like PDFs, plain texts, and structured data for importing and exporting transcripts
	\end{itemize}
    \item  \textbf{ Communications Interfaces}
	\begin{itemize}
		 \item    Datawill be transmitted securely over the Internet using HTTPS encryption. The system
 will comply with LACPD's communications protocols for data integrity and
 confidentiality
	\end{itemize}
    \item  \textbf{ Memory}
	\begin{itemize}
		 \item    Efficiently manage memory usage, particularly during data processing phrases.
 Dynamically allocate resources based on system demands
	\end{itemize}
    \item  \textbf{Operations}
	\begin{itemize}
		 \item    Operated by the LACPD IT department with guidelines in place for updates, security
 checks, and user support. Regular maintenance will be held to ensure proper functionality
 and the latest support.
	\end{itemize}
    \item  \textbf{ Site Adaptation Requirements}
	\begin{itemize}
		 \item    Thedeployment will be tailored to the LACPD operational environment such as a variety
 of cases, workflow, training management, and data backup and recovery
		\item Asthedepartment continues to grow, further updates will be pushed to adhere to different
 changes.
	\end{itemize}
\end{itemize}

\subsection{Product Functions}
\begin{itemize}
  \item  Real-time Transcript Ingestion and optional OCR Processing: Allow upload of court
 transcripts in various formats (PDF, DOCX), converting non-searchable text into searchable
 format where needed. OCR Processing will be used as a failsafe
	\begin{itemize}
	\item     If the file includes non-machine-readable text (e.g., scanned transcript), OCR is applied to extract content.
	\end{itemize}

  \item TextAnalysis, Entity Extraction, and Misconduct Detection: Utilize NLP and generative AI
 models to identify and extract key entities (e.g., officer names, badge numbers, locations,
 dates, charges). The system will analyze transcripts to detect potential patterns of misconduct
 or inconsistencies within and across transcripts

  \item Relationship Linking and Case Correlation: Use machine learning algorithms to identify
 links between cases based on shared attributes and contextual similarities, presenting
 attorneys with visual case connections.

  \item  Multi-Level Flagging System: Generate flags on transcripts and highlight key findings.
 Flagging will incorporate color-coded indicators to convey levels of confidence in
 misconduct detection (e.g., red for high-confidence flags, yellow for moderate).
\end{itemize}

\subsection{User Characteristics} 
The intended users of the system include:
\begin{itemize}
  \item Attorneys, Paralegals, Legal Assistants
	\begin{itemize}
	\item     Attorney
		\begin{itemize}
		\item    Varying levels of experience from recent law school graduates to years of practice
		\item  Familiar with databases and case management software
		\end{itemize}
	\item  Paralegals and Legal Assistants:
		\begin{itemize}
		\item     Education level often includes an associate's degree or bachelor's degree. A certificate in paralegal studies would be a bonus
		\item Experience typically possesses some years in legal environments, such as document management, case preparation, and client interaction
		\item  Technical expertise may be familiar with legal documentation and basic software tools
		\item They mayhavespecialized training in investigative techniques and are comfortable with technological tools.
		\end{itemize}
	\end{itemize}
  \item LACPD IT Administrators: Utilize the management portal to ensure the proper state and
 function of the software system
	\begin{itemize}
	\item      LACPD IT Administrators
		\begin{itemize}
		\item     Education level often carries a degree in information technology or computer science
		\item Experience in managing software systems and security performance
		\item Technical Expertise includes a high level of technical knowledge with proficiency in software management, system integration, and data security
		\end{itemize}
	\end{itemize}
\end{itemize}

\subsection{General Constraints}
\begin{itemize}
  \item The system must comply with LACPD’s data privacy and security regulations.
  \item AWS cloud services will be the primary platform for all data processing and storage.
  \item The system must support multi-user access with role-based permissions.
  \item The system should have a 24 hour RPO and a <= 6 hour RTO
\end{itemize}

\subsection{Assumptions and Dependencies}
\textbf{ Availability of Training Data (Transcripts)}
\begin{itemize}
  \item Thedevelopment and effectiveness of machine learning models for this project rely on
 continuous access to accurate, high-quality transcripts. Court transcripts, as the primary data
 source, must be:
	\begin{itemize}
		\item   Timely and accurately transcribed, capturing recent cases and maintaining consistent language patterns
		\item  Formatted uniformly to support reliable processing across models
		\item Extensive enough to enable effective model training for accurate insights
	\end{itemize}
  \item  Any issues, such as delays or formatting inconsistencies, could significantly impact the
 model’s performance. Therefore, uninterrupted access to high-quality transcripts is critical to
 project success
\end{itemize}

\textbf{ Integration with LACPD’s Case Management System (CMS) and Other Databases}
\begin{itemize}
  \item  Successful implementation requires seamless integration with LACPD’s existing CMS and
 related databases, allowing the system to retrieve relevant data, process it, and store results.
 Key integrations include:

	\begin{itemize}
		\item    Bi-directional data exchange with LACPD’s CMS to enable real-time updates, facilitating immediate availability of flagged information.
		\item   Secure APIs for data transfers to maintain data integrity and confidentiality across systems.

	\end{itemize}
  \item Any interruptions or updates to LACPD’s existing systems may impact the functionality and
 performance of the machine learning models. Ensuring compatibility and establishing
 reliable communication protocols with these systems is essential to maintain a seamless
 workflow for the legal team
\end{itemize}

\textbf{  AWS Infrastructure and Service Availability}
\begin{itemize}
  \item  AWS infrastructure will support scalable, flexible, and secure deployment for model training,
 inference, and data storage. The system will rely on specific AWS services, such as
	\begin{itemize}
		\item    SageMaker: For model training, inference, and deployment
		\item  Box.com: For secure storage of transcript data and associated metadata.
		\item  Comprehend: For Natural Language Processing and key entity extraction.
	\end{itemize}
  \item The assumption is that AWS services will remain accessible, reliable, and cost-effective
 throughout the project lifecycle. Any changes or interruptions to these services, such as
 unexpected downtimes or cost increases, may impact project timelines, budget, and
 outcomes.
\end{itemize}

\textbf{  Data Security and Compliance Requirements}
\begin{itemize}
  \item   Thesystem must comply with LACPD’s data privacy and security policies, as well as
 applicable legal and regulatory standards. This includes:
	\begin{itemize}
		\item    Adherence to secure data handling practices within AWS, such as encryption at rest and in transit, to protect sensitive case information.
		\item Role-based access control to ensure that only authorized personnel can access the system’s data and functions

	\end{itemize}
  \item Maintaining this security and compliance is critical to building a trusted system for legal use,
 and any changes in policy may require updates to the system’s design or architecture
\end{itemize}

\section{Requirements}
\subsection{Functional Requirements}

\textbf{ Transcript Upload and Storage}
\begin{itemize}
  \item \textbf{Description} The system shall allow users to upload court transcripts in supported formats
 (PDF, DOCX). OCR will convert any non-searchable document text into a searchable format
  \item \textbf{Requirement}  The system shall store all uploaded and processed transcripts in Box.com with
 encryption, ensuring secure access for authorized personnel.
\end{itemize}

\textbf{ Text Analysis, Key Entity Extraction, and Misconduct Detection}
\begin{itemize}
  \item \textbf{Description} The system shall utilize NLP models to extract key entities within transcript
 text (e.g., officer names, badge numbers, locations, charges) and detect misconduct patterns
 or inconsistencies, automatically flagging critical sections for legal review
  \item \textbf{Requirement} AWS Comprehend or an equivalent NLP model will handle entity extraction
 and misconduct detection. Results will be stored in a structured, searchable format accessible
 to authorized users.
\end{itemize}

\textbf{  Case Linking and Relationship Detection}
\begin{itemize}
  \item \textbf{Description} The system shall identify and link related cases by analyzing shared attributes
 or contextual similarities, providing visual connections for easier analysis and tracking.
  \item \textbf{Requirement}  The system shall employ clustering or similarity algorithms to detect and
 present case relationships to users through an intuitive interface, facilitating cross-case
 analysis.
\end{itemize}

\textbf{ Multi-Level Flagging System}
\begin{itemize}
  \item \textbf{Description} The system shall generate color-coded flags within transcripts, indicating
 varying confidence levels in detected issues (e.g., red for high-confidence flags, and yellow
 for moderate confidence).
  \item \textbf{Requirement}   Flags shall be generated and applied automatically by the system based on
 NLPanalysis results, with options for users to export flagged sections or entire transcripts in
 PDF or CSVformats for further review and case reporting.
\end{itemize}

\subsection{External Interface Requirements}


\textbf{  Dashboard}
\begin{itemize}
  \item \textbf{Purpose} Central hub for accessing and managing case information
  \item \textbf{Key features}   
	\begin{itemize}
		\item  Overview of Active Cases: Display a list of ongoing cases with quick links to associated
 documents
		\item Recent Activity Feed: Show recently uploaded, reviewed, or flagged documents
		\item  Notifications Panel: Highlight critical updates, such as flagged issues, pending uploads,
 or incomplete tasks.
		\item Search Bar: Provide a global search tool for locating cases, transcripts, or flagged
 information.
		\begin{itemize}
			\item    Adherence to secure data handling practices within AWS, such as encryption at rest and in transit, to protect sensitive case information.
			\item Role-based access control to ensure that only authorized personnel can access the system’s data and functions
		\end{itemize}
	\end{itemize}
\end{itemize}

\textbf{  Case Explorer}
\begin{itemize}
  \item \textbf{Purpose} Allow users to manage and organize case-related files
  \item \textbf{Key features}   
	\begin{itemize}
		\item   File Directory: Organized folder structure by case ID, client name, or metadata tags.
		\item Filter Options: Enable sorting by date, document type, flagged issues, or entity mentions.
		\item  BulkActions: Allow multiple files to be uploaded, moved, or annotated simultaneously.
	\end{itemize}
  \item \textbf{Notes}   
	\begin{itemize}
		\item    Adherence to secure data handling practices within AWS, such as encryption at rest and in transit, to protect sensitive case information.
		\item Role-based access control to ensure that only authorized personnel can access the system’s data and functions
	\end{itemize}

\end{itemize}

\textbf{ Document Viewer}
\begin{itemize}
  \item \textbf{Purpose}  Provide a workspace for attorneys to analyze and interact with case documents.
  \item \textbf{Key features}   
	\begin{itemize}
		\item  Export Options: Allow attorneys to download annotated or flagged versions of documents.
		\item Entity Insights: Present identified entities (e.g., officer names, badge numbers) in an interactive side panel.
		\item Annotations: Enable users to add notes, highlight key sections, or bookmark pages
		\item   Search Within Document: Allow users to search for specific terms or phrases in the transcript
		\item Split-Screen View:
		\begin{itemize}
			\item     Left Panel: Display the full transcript or case document
			\item Right Panel: Highlight flagged content (e.g., misconduct patterns, inconsistencies).
		\end{itemize}
	\end{itemize}
  \item \textbf{Notes}   
	\begin{itemize}
		\item    Version Control: Include a feature to track changes and view previous versions of annotated documents
	\end{itemize}
\end{itemize}

\textbf{   Advanced Search and Filter}
\begin{itemize}
  \item \textbf{Purpose}  Help users find specific information quickly
  \item \textbf{Key features}   
	\begin{itemize}
		\item  Global Search: Search across all cases for transcripts, flagged issues, or specific keywords.
		\item SavedSearches: Let users save frequent searches for future use.
		\item Advanced Filters: Narrow down results based on
		\begin{itemize}
			\item     Daterange.
			\item  Document type (e.g., transcript, evidence).
			\item Metadata (e.g., client name, case ID).
		\end{itemize}
	\end{itemize}
\end{itemize}

\textbf{  Insights and Reports}
\begin{itemize}
  \item \textbf{Purpose} Simplify the process of adding documents to the system
  \item \textbf{Key features}   
	\begin{itemize}
		\item   Drag-and-Drop Upload: Allows users to easily drag and drop files for upload
		\item  File Validation: Immediately check file type, size, and format compatibility.
		\item  Progress Indicator: Show upload status with estimated time to completion
		\item  Error Handling: Notify users of issues (e.g., file too large) with actionable guidance.
	\end{itemize}
\end{itemize}

\textbf{  Upload Interface}
\begin{itemize}
  \item \textbf{Purpose} Central hub for accessing and managing case information
  \item \textbf{Key features}   
	\begin{itemize}
		\item  Overview of Active Cases: Display a list of ongoing cases with quick links to associated
 documents
		\item Recent Activity Feed: Show recently uploaded, reviewed, or flagged documents
		\item  Notifications Panel: Highlight critical updates, such as flagged issues, pending uploads,
 or incomplete tasks.
		\item Search Bar: Provide a global search tool for locating cases, transcripts, or flagged
 information.
		\begin{itemize}
			\item    Adherence to secure data handling practices within AWS, such as encryption at rest and in transit, to protect sensitive case information.
			\item Role-based access control to ensure that only authorized personnel can access the system’s data and functions
		\end{itemize}
	\end{itemize}
\end{itemize}

\textbf{  Notifications and Alerts}
\begin{itemize}
  \item \textbf{Purpose}  Keep users informed about system activity and issues.
  \item \textbf{Key features}   
	\begin{itemize}
		\item   Success Notifications: Confirm successful actions (e.g., "File uploaded successfully").
		\item Error Alerts: Notify users of system errors with clear instructions (e.g., "Connection lost.
 Please try again.").
		\item   Reminders: Prompt users about pending actions or deadlines (e.g., "Review flagged
 documents in Case number 12345.").

	\end{itemize}
\end{itemize}

\textbf{  User Settings and Preferences}
\begin{itemize}
  \item \textbf{Purpose }Allow customization to enhance user experience.
  \item \textbf{Key features}   
	\begin{itemize}
		\item  Profile Settings: Update user details, such as name or role
		\item  Accessibility Options: Adjust font size, color contrast, or language preferences.
		\item   Notification Preferences: Choose how and when to receive alerts (e.g., email or in-app)
	\end{itemize}
\end{itemize}




\subsection{User Interfaces}

\begin{itemize}
  \item The User Interface of the LACPD Transcript Analysis System is crucial for ensuring that
 users, regardless of their technical expertise can easily utilize and interact with the system to
 perform their tasks efficiently and accurately. This section will specify the requirements for
20 the system's user interfaces, including configuration details, content organization, and guidelines for optimizing user experience.
	\begin{itemize}
		\item   Requirements of Each Interface Between the Software Product and Its Users
		\begin{itemize}
			\item  Dashboard: The system's primary screen should be a dashboard that provides users with an overview of their current tasks and cases. The dashboard will include:
			\begin{itemize}
				\item  A case search bar for navigation and retrieval of court transcripts
				\item Visualizations of case relations (graphs, charts, etc)
				\item Summaryofrecently accessed cases and documents.
				\item Quickaccess buttons for common tasks such as “Analyze Transcripts”, “Download Report”, and “View Case History”.
			\end{itemize}
		\end{itemize}
		\begin{itemize}
			\item   Search Results and Case View: When a user performs a search, the result should be displayed in an easy-to-read format, showing key metadata such as:
			\begin{itemize}
				\item   Case ID
				\item Case Title
				\item Case Date
				\item A summary of the transcripts
				\item  The user can click on individual results for more detailed information including full analysis.
			\end{itemize}
		\end{itemize}
		\begin{itemize}
			\item Analysis Results: When a transcript is processed, the results should be displayed in a structured format that includes:
			\begin{itemize}
				\item  Key entities (e.g. names, locations, charges, etc.) highlighted within the transcript
				\item Categorized segments of the transcript (e.g. introduction, witness testimony, verdict, etc)
				\item  Confidence score that indicates the reliability of the model analysis.
				\item  Interactive graphs or charts displaying relationships between entities (e.g. the relationship between defendants, witness, and location.)
			\end{itemize}
		\end{itemize}
		\begin{itemize}
			\item  Reports and Exports: The system will allow users to download reports summarizing the analysis results. Reports will include:
			\begin{itemize}
				\item  Full court transcripts with analysis annotations.
				\item Summary tables of key information.
				\item Charts for insights
			\end{itemize}
		\end{itemize}
		\begin{itemize}
			\item   Error or Confirmation Notification: When users interact with the system, feedback involving error and confirmation notifications will be given to help users navigate the user interface.
			\begin{itemize}
				\item  Error message (e.g. “Invalid case ID”)
				\item Success message (e.g. “Report generated successfully!”)
			\end{itemize}
		\end{itemize}
		\begin{itemize}
			\item  Navigation and Interaction: The interface should provide navigation menus with clear labels to access various sections of the application, such as:
			\begin{itemize}
				\item  Case Search
				\item Transcript Analysis
				\item Reports
				\item  Settings/Configuration (admin/IT staff)
			\end{itemize}
		\end{itemize}
		\begin{itemize}
			\item  A back button will allow users to return to the previous screen or dashboard.
		\end{itemize}
		\begin{itemize}
			\item  Content Formatting
			\begin{itemize}
				\item   Reports should be organized in a readable manner and format. Key fields such as case numbers, dates, and extracted entities will be presented in a tabular format where filtering and sorting will be an option	
			\end{itemize}
		\end{itemize}
	\end{itemize}
  \item General Interface Characteristics
	\begin{itemize}
		\item   Consistency: The software shall maintain a consistent look and feel across all screens and modules, including uniform color schemes, fonts, and button styles
		\item   Responsiveness: The interface shall be responsive and adapt to various screen sizes and resolutions.
	\end{itemize}
  \item Optimizing User Interaction
	\begin{itemize}
		\item   Do’s
		\begin{itemize}
			\item Provide Immediate Feedback: The system shall offer immediate visual or auditory feedback in response to user actions (e.g., button clicks, documentation download)
			\item  UseFamiliar Icons and Symbols: Commonly recognized icons shall be used to represent standard actions
		\end{itemize}

		\item   Don’t
		\begin{itemize}
			\item   Information Overload: The interface shall not display unnecessary information that may overwhelm the user.
		\end{itemize}
	\end{itemize}

\end{itemize}

\subsection{Hardware Interfaces}
\begin{itemize}
  \item The system shall support interaction with AWS cloud hardware for model training and deployment
\end{itemize}

\subsection{ Software Interfaces}
 \textbf{  Amazon EC2}\\
 Name: Amazon Elastic Compute Cloud\\
 Mnemonic: EC2\\
 Specification Number: AWS-EC2-001\\
 Version Number: Latest available at project initiation\\
 Source: AWS

 \textbf{  Purpose:}\\
 EC2 instances provide computational resources to support model inference, data processing, and
 other computationally intensive tasks as required by the system.

 \textbf{  Interface Definition:}\\
 Input: Compute job requests (e.g., for data preprocessing or model inference).\\
 Output: Processed data passed to storage services like S3 or directly to SageMaker for further
 analysis.\\

 Format Reference: AWS EC2 API (AWS Developer Documentation) for starting, stopping, and
 monitoring instances.
\begin{itemize}
  \item AWS Services: The system shall use AWS SageMaker for NLP model training and
 inference, S3 for secure storage of transcripts, and EC2 for computational needs. Data
 processed on AWS will be encrypted and accessible only to authenticated users.
  \item The system shall use AWS SDKs to interact with cloud services and APIs for external
 communication.
\item The system shall use Box.com to store testimony transcripts and other related documents
 securely.
\item The system shall integrate with Box.com for secure cloud storage and management of court
 transcripts and related documents.
\end{itemize}

\subsection{Communications Interfaces}
\begin{itemize}
  \item The system shall use secure HTTPS protocols for all data transfers and communications
 between the client and the web server
  \item  The system shall use AWS’s encryption and associated security services to ensure the data
 stays secure throughout the data pipeline including storage

\end{itemize}

\section{Requirements Specification}
See Section 3 for detailed breakdown. Structured JSON/CSV export capabilities, graph databases, and referential integrity required.

\subsection{Functional Requirements}
\subsubsection{Transcript Upload and Storage}
\begin{itemize}
  \item  The system shall allow users to upload court transcripts in supported formats (PDF, DOCX)
  \item  The system may apply OCR to convert non-machine-readable transcripts into searchable text.
\item The system shall store all uploaded transcripts securely in Box.com, with encryption at rest
 and in transit
\item The system shall associate uploaded transcripts with case metadata, such as case ID, date,
 and document type, for efficient organization and retrieval
\end{itemize}

\subsubsection{Text Analysis, Entity Extraction, and Misconduct Detection}
\begin{itemize}
  \item   The system shall utilize NLP models to identify key entities (e.g., officer names, badge
 numbers, locations) within transcripts.
  \item  The system shall detect and flag potential misconduct patterns, including inconsistencies
 within transcripts and related cases.
\item  The system shall generate structured, searchable outputs for all analysis results, including
 entity lists and flagged sections.
\item The system shall calculate and display confidence scores for flagged findings
\end{itemize}

\subsubsection{Case Linking and Relationship Detection}
\begin{itemize}
  \item   The system shall analyze shared attributes or contextual similarities between cases to identify
 relationships.
  \item  The system shall present case linkages visually, using graphs or relationship diagrams, to
 highlight connections and shared entities
\item   The system shall allow users to filter relationship results by criteria such as case date,
 involved parties, or flagged patterns
\end{itemize}

\subsubsection{Multi-Level Flagging System}
\begin{itemize}
  \item   The system shall assign color-coded flags to indicate levels of confidence in detected
 misconduct patterns (e.g., red for high confidence, yellow for moderate).

  \item  The system shall allow users to view and manage flagged sections within the transcript
 viewer
\item  The system shall enable the export of flagged findings and associated metadata in formats
 such as PDF or CSV
\end{itemize}

\subsection{External Interface Requirements}
\subsubsection{Dashboard}
\begin{itemize}
  \item  The system shall provide a dashboard displaying active cases, recent activity, and
 notifications.
  \item  The system shall include a global search bar to locate cases, transcripts, or flagged findings.

\end{itemize}

\subsubsection{Case Explorer}
\begin{itemize}
  \item   The system shall allow users to organize and manage case-related documents using folder
 structures.
  \item  The system shall support filtering and sorting by attributes such as document type, upload
 date, and flagged status.
\end{itemize}

\subsubsection{Document Viewer}
\begin{itemize}
  \item    The system shall provide a split-screen view with the transcript on one side and flagged
 content on the other.
  \item   The system shall support annotations, bookmarking, and interactive exploration of entities.
\end{itemize}

\subsubsection{ Advanced Search and Filter}
\begin{itemize}
  \item   The system shall allow users to perform global searches across all cases using keywords or
 metadata.

  \item   The system shall support advanced filters such as date range, document type, or flagged
 content status.
\end{itemize}


\subsubsection{ Insights and Reports}
\begin{itemize}
  \item    The system shall generate summary reports of flagged misconduct, identified entities, and
 key case insights.
  \item  The system shall allow export of reports in multiple formats, including PDF and Excel.
\end{itemize}


\subsubsection{Upload Interface}
\begin{itemize}
  \item  The system shall support drag-and-drop file uploads and manual file selection.
  \item  The system shall provide real-time validation feedback for file type and size
\end{itemize}


\subsubsection{ Notifications and Alerts}
\begin{itemize}
  \item   The system shall notify users of key events, such as flagged document readiness or pending
 tasks
  \item  The system shall provide error messages with actionable guidance for troubleshooting.
\end{itemize}

\subsection{Logical Database Requirements}
\begin{itemize}
  \item The system shall maintain a database linking transcripts, flagged findings, and associated metadata
  \item  Thesystem shall enforce referential integrity between cases, documents, flags, and metadata tables
\item The system shall store relationships between cases using graph structures for efficient retrieval and visualization
\item  The system shall support indexing for search capabilities on attributes such as case ID, entity type, and flagged content.
\end{itemize}

\subsection{Design Constraints}
\begin{itemize}
  \item The system shall comply with AWS’s encryption standards for data security, including AES-256 for data at rest and TLS for data in transit.
  \item The system shall support real-time processing of large datasets, scaling horizontally using AWS services like SageMaker and EC2.
\item The system shall adhere to role-based access control (RBAC) to restrict access based on user roles and permissions.
\item  The system shall ensure compatibility with existing LACPD tools, including Box.com and Salesforce.
\item The system shall achieve a recovery time objective (RTO) of less than or equal 6 hours and a recovery point objective (RPO) of 24 hours
\end{itemize}

\section{Other Nonfunctional Requirements}
\subsection{Performance Requirements}
\begin{itemize}
  \item UserCapacity: The system is expected to support multiple simultaneous users, including
 attorneys, investigators, and administrative staff. The exact number of supported users under
 normal and peak conditions will be determined after the client's consultation and a detailed
 workload analysis.
  \item Response Time: The system should provide acceptable response times for user actions,
 including uploading documents, retrieving flagged findings, and performing searches.
 Specific response time thresholds will be defined based on user feedback and operational
 requirements gathered during implementation
\item Data Processing: The system shall efficiently handle data upload, processing, and retrieval
 tasks. Detailed processing speed and throughput benchmarks will be established after further
 testing with realistic transcript datasets.
\item Scalability: The system shall be designed to scale horizontally to accommodate increasing
 user demand and data volumes. Specific scalability targets will be refined after the client's
 input and evaluation of operational needs
\end{itemize}

\subsection{Safety Requirements}
\begin{itemize}
  \item  The system shall provide safeguards to ensure the integrity and security of sensitive legal
 data.
  \item In the event of a system failure, the system shall implement measures to preserve data
 integrity and enable recovery within the defined RTO of less than or equal to 6 hours.
  \item  The system shall include fail-safe mechanisms to prevent accidental data loss or corruption
 during processing or user operations.
\item The system shall adhere to industry standards for legal software to prevent unauthorized
 access, misuse, or data breaches that could harm clients or attorneys.
\end{itemize}

\subsection{Security Requirements}
\begin{itemize}
  \item The system shall enforce role-based access control (RBAC) to ensure that only authorized
 personnel can access sensitive data and perform specific operations
  \item All data at rest shall be encrypted using AES-256, and all data in transit shall be encrypted
 using TLS protocols
\item The system shall require multi-factor authentication (MFA) for all users with administrative
 roles.
\item  Logs of all user activities, including access, modifications, and uploads, shall be maintained
 for auditing purposes
\item The system shall comply with legal and regulatory standards, such as GDPR and CCPA, as
 well as the LACPD's internal data security policies.
\item   Regular security audits and penetration testing shall be conducted to identify and address
 vulnerabilities.
\end{itemize}

\subsection{Quality Attributes}
\begin{itemize}
  \item Adaptability: The system shall be adaptable to changes in legal workflows and easily
 configurable to accommodate updates in legal standards or organizational needs
  \item Availability: The system shall be operational 24/7 with an uptime of at least 99.5%, except
 during scheduled maintenance.
\item  Reliability: The system shall ensure accurate and consistent results across all transcript
 analyses, with error rates kept below a predefined threshold to be determined after initial
 testing
\item Maintainability: The system shall be modular to facilitate updates and maintenance,
 ensuring minimal downtime and impact on users during upgrades.
\item  Interoperability: The system shall integrate seamlessly with Box.com, AWS, and other
 LACPD tools, ensuring data exchange without compatibility issues.
\item  Usability: The user interface shall be intuitive and easy to navigate, ensuring minimal
 training requirements for end-users.
\end{itemize}

\subsection{Business Rules}
\begin{itemize}
  \item Attorneys and investigators shall have access only to case files assigned to them, as per
 LACPD’s case management policies.
  \item Administrative staff shall have read-only access to case documents unless explicitly granted
 editing permissions
  \item All flagged findings shall require manual review by authorized legal personnel before
 inclusion in court submissions
\item Data retention policies shall comply with LACPD’s guidelines, ensuring that case files are
 archived or deleted based on predefined timelines
\item  External analysts and consultants shall be granted temporary access to specific cases or
 reports, with access automatically revoked after a set duration.
\item All system changes or updates shall require approval from LACPD IT administrators before
 deployment
\end{itemize}

\section{Legal and Ethical Considerations}
 This project addresses key legal and ethical challenges, including user privacy, potential harm,
 intellectual property, and system security. Adhering to the ACM Code of Ethics and Professional
 Conduct, our approach ensures compliance with ethical standards and legal requirements.\\
 User Privacy:\\
 Handling sensitive court transcripts requires compliance with principles 1.6 (Respect
 Privacy) and 1.7 (Honor Confidentiality). We minimize data collection, anonymize records
 where feasible, and implement encryption and role-based access controls to secure sensitive
 information. The system complies with GDPR, CCPA, and LACPD-specific privacy standards,
 ensuring users’ transparency and control over their data.\\
 Avoiding Harm:\\
 To mitigate risks like biased analyses or misuse, we follow principle 1.2 (Avoid Harm) by
 testing and validating NLP models to reduce errors and biases. Security protocols, including
 HTTPS and AWSservices, protect against unauthorized access. Flagged results support
 attorneys’ decisions without replacing their expertise.\\
 Intellectual Property:\\
 The intellectual property rights for the LACPD Transcript Analysis System will reside
 exclusively with the Los Angeles Public Defender’s Office (LACPD). All software components,
 including the Natural Language Processing (NLP) models, machine learning pipelines, and user
 interface designs, are developed specifically for LACPD's use and are the sole property of the
 organization. Additionally, any case-related data, transcripts, or other sensitive information
 processed within the system remains under the ownership of LACPD, and the system is designed
 not to retain or use this data beyond its intended scope. To protect the integrity of this intellectual
 property, the system employs secure communication protocols, such as HTTPS, ensuring that
 data exchanged between LACPD systems and the application remains confidential and secure.\\
 System Security:\\
 Aligned with Principle 2.9, the system leverages AWS services for scalable, reliable, and
 secure deployment. Regular updates and thorough security audits are conducted to ensure
 compliance with evolving industry standards, safeguarding the system against potential
 vulnerabilities and maintaining the trust of stakeholders.
 The system prioritizes privacy and minimizes potential harm by implementing robust data
 protection measures, respecting intellectual property rights, and adhering to legal and ethical
 guidelines. By integrating these practices, the system delivers actionable insights while ensuring
 data security, user confidentiality, and ethical compliance
\appendix
\section{Appendix A: Glossary}
See document content for full definitions of: LACPD, NLP, OCR, AWS, RBAC, MFA, RTO, RPO, GDPR, CCPA.

\section{Appendix B: System Models}
 This appendix will contain the models that visually represent the system's design and its
 interactions. For this project, include:
\begin{itemize}
  \item Data Flow Diagram (DFD)  Visual representation of how court transcripts flow through the
 system, including upload, OCR processing, analysis, and output.
  \item Entity-Relationship Diagram (ERD): Show relationships between core database entities,
 such as: 
	\begin{itemize}
		\item Transcripts
		\item Cases
		\item Users
		\item Flags
	\end{itemize}
  \item State Transition Diagram: Define how documents move between states (e.g., Uploaded-
 Processed- Flagged- Reviewed)
  \item System Architecture Diagram: High-level diagram illustrating AWS services (e.g., S3,
 SageMaker, Comprehend) and how they interact with the system.
\end{itemize}
 Each of these models should be created using tools like Lucidchart, Visio, or similar, and
 attached here for clarity

\section{Appendix C: To Be Determined List}
 Collect a numbered list of the TBD (to be determined) references that remain in the SRS so they
 can be tracked to closure
\begin{itemize}
  \item TBD-1: Define exact user response times
  \item TBD-2: Finalize max user capacity
  \item TBD-3: Acceptable transcript upload latency
  \item TBD-4: Salesforce integration confirmation
  \item TBD-5: NLP model accuracy benchmarks
\end{itemize}

\end{document}
